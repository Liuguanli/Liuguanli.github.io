% !TEX program = xelatex

\documentclass{resume}
%\usepackage{zh_CN-Adobefonts_external} % Simplified Chinese Support using external fonts (./fonts/zh_CN-Adobe/)
%\usepackage{zh_CN-Adobefonts_internal} % Simplified Chinese Support using system fonts

\begin{document}
\pagenumbering{gobble} % suppress displaying page number

\name{Guanli Liu}

\basicInfo{
  \email{liuguanli22@gmail.com} \textperiodcentered\ 
  \phone{(+61) 450075953} \textperiodcentered\ 
  \linkedin[Guanli Liu]{https://www.linkedin.com/in/guanli-liu-11353058/}}

I am a third-year PhD student at the University of Melbourne. My research interest is using machine learning
technology to solve database related problems, for example, index building. Before UoM, I was a senior
researcher and developer in Baidu Inc. in Beijing, China.


\section{\faGraduationCap\ Education}
\datedsubsection{\textbf{The University of Melbourne}, Melbourne, Australia}{2019 -- Present}
\textit{PhD Candidate} in Computing and Information System, expected December 2022
\datedsubsection{\textbf{Northeastern University}, Shenyang, China}{2013 -- 2015}
\textit{M.S.} in Computer Technology
\datedsubsection{\textbf{Northeastern University}, Shenyang, China}{2009 -- 2013}
\textit{B.S.} in Software Engineering


\section{\faFilesO\ Papers}
\begin{itemize}
 \item Guanli Liu, Lars Kulik, Xingjun Ma, Jianzhong Qi. A Lazy Approach for Efficient Index Learning. arXiv preprint arXiv:2102.08081. \\ 
{source code: }{https://github.com/Liuguanli/ModelReuse}
  \item Jianzhong Qi (supervisor), Guanli Liu, Christian S. Jensen, and Lars Kulik. Effectively Learning Spatial Indices. PVLDB, 13(11): 2341-2354, 2020. \\
{source code: }{https://github.com/Liuguanli/RSMI}
  \item Yu Gu (supervisor), Guanli Liu, Jianzhong Qi, Hongfei Xu, Ge Yu, and Rui Zhang. The Moving K Diversified Nearest Neighbor Query. TKDE, 28(10): 2778-2792, 2016.
\end{itemize}

\section{\faUsers\ Experience}

\datedsubsection{\textbf{The University of Melbourne}}{Aug. 2019 -- Present}
\role{Tutor}{Comp90018 (Mobile Computing)}
%Brief introduction: xxx
\begin{itemize}
  \item Android development
    \item Project guidance
  \item Assignment marking
\end{itemize}
\role{Tutor}{Comp90041 (Programming and Software Development)}
%Brief introduction: xxx
\begin{itemize}
  \item Java development
    \item Assignment marking
\end{itemize}

\datedsubsection{\textbf{Baidu Inc.} Beijing, China}{2015 -- 2017}
\role{Senior Researcher and Developer}{Android developer}
 Instant Messaging application development based on Android system. Key contributions are as follows:
\begin{itemize}
  \item Message: Message recall, Message deletion, Red pocket Message, Contact Message
  \item Projects: QR code recognition, file transfer assistant, splash, global message search 
  \item Voice assistant: Voice recognition SDK integration, send message, order meeting room
\end{itemize}


\datedsubsection{\textbf{Neusoft } Shenyang, China}{Jun. 2012 -- Sep. 2012}
\role{Software Engineer}{Java Web development}

\begin{itemize}
  \item A team work (1/6) of administrative system of the logistics 
\end{itemize}

% Reference Test
%\datedsubsection{\textbf{Paper Title\cite{zaharia2012resilient}}}{May. 2015}
%An xxx optimized for xxx\cite{verma2015large}
%\begin{itemize}
%  \item main contribution
%\end{itemize}

\section{\faCogs\ Skills}
\begin{itemize}[parsep=0.5ex]
  \item Programming Languages: Java > Python == C++
%  \item Platform: Linux
  \item Development: Java Web, Java, Android, SQL, LaTex, TensorFlow, PyTorch
%    \item Development: Java Web, Java, Android, SQL, LaTex, TensorFlow, PyTorch
%  \item Development: Java Web, Java, Android, SQL, LaTex, TensorFlow, PyTorch
  
\end{itemize}

\section{\faHeartO\ Honors and Awards}
\datedline{performance appraisal in Baidu \textit{Top 20\%}}{2016}
\datedline{Outstanding master student \textit{Top 5\%}}{2014}
\datedline{\textit{\nth{3} Prize} prize of American college students' mathematical modelling contest}{2011}
\datedline{\textit{\nth{3} Prize} prize of Google Android application development in China}{2011}

\section{\faInfo\ Miscellaneous}
\begin{itemize}[parsep=0.5ex]
%  \item Blog: http://your.blog.me
  \item GitHub: https://github.com/Liuguanli
  \item Languages: English - Fluent, Mandarin - Native speaker
\end{itemize}

%% Reference
%\newpage
%\bibliographystyle{IEEETran}
%\bibliography{mycite}
\end{document}
