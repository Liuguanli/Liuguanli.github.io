\documentclass{resume}
% \usepackage{zh_CN-Adobefonts_external} % Simplified Chinese Support using external fonts (./fonts/zh_CN-Adobe/)
% \usepackage{zh_CN-Adobefonts_internal} % Simplified Chinese Support using system fonts

\newcommand{\SoftwareEngineer}{Software Engineer}
\newcommand{\DataScientist}{Data Scientist}
\newcommand{\ResearchFellow}{Postdoctoral Research Fellow}



\usepackage{xcolor}
\begin{document}
\pagenumbering{gobble} % suppress displaying page number

\name{Guanli Liu}

\basicInfo{
  \email{guanlil1@unimelb.edu.au} \textperiodcentered\ 
  % \email{liuguanli22@gmail.com} \textperiodcentered\ 
  \phone{(+61) 450075953} \textperiodcentered\ 
  \linkedin[Guanli Liu]{https://www.linkedin.com/in/guanli-liu-11353058/} \textperiodcentered\ 
  \github[https://github.com/Liuguanli]{https://github.com/Liuguanli}
  }

% \section{\faUser\ \textbf{Profile}}
% As a newly minted PhD graduate specializing in the application of machine learning to build database indices (\textbf{AI4DB}), I possess robust data analysis and machine learning skills.





% \section{\faGraduationCap\ \textbf{Education}}
% \datedsubsection{\textbf{The University of Melbourne}, Melbourne, Australia}{Jun. 2019 -- June 2023} 
% \textit{PhD Candidate} in Computing and Information System (GPA: H1)
% \datedsubsection{\textbf{Northeastern University}, Shenyang, China}{Aug. 2013 -- Jun. 2015}
% \textit{M.S.} in Computer Technology (GPA: 3.5 / Top 10\%)
% \datedsubsection{\textbf{Northeastern University}, Shenyang, China}{Aug. 2009 -- Jun. 2013}
% \textit{B.Eng.} in Software Engineering (GPA: 3.4 / Top 20\%)

\section{\faUser\ \textbf{Profile}}
Data scientist with a PhD in AI4DB, specializing in probabilistic modeling, GenAI-assisted analytics, and production-grade experimentation. I translate messy behavioral data into ranking, risk, and recommendation models for teams in Australia and the US.

\section{\faCogs\ \textbf{Analytics \& AI Skills}}
\begin{itemize}[parsep=0.35ex]
  \item \textbf{Modeling:} Causal inference, gradient boosting, sequence models, RL for tuning, LLM/RAG prompt engineering
  \item \textbf{Tooling:} Python (pandas, PyTorch, TensorFlow, scikit-learn), BigQuery, dbt, SQL, Spark, Airflow
  \item \textbf{Experimentation:} A/B testing, Bayesian optimization, feature stores, ML monitoring, bias/fairness analysis
  \item \textbf{GenAI:} GPT-4/Claude APIs, embeddings, retrieval pipelines, safety filters, synthetic data generation
  \item \textbf{Visualization \& Storytelling:} Looker, Tableau, Plotly, executive dashboards, product requirement writing
\end{itemize}

\section{\faGraduationCap\ \textbf{Education}}
\datedsubsection{\textbf{PhD} in Computer Science, \textit{The University of Melbourne}, Australia}{2019 -- 2023}
\datedsubsection{\textbf{M.S.} in Computer Technology, \textit{Northeastern University}, China}{2013 -- 2015}
\datedsubsection{\textbf{B.Eng.} in Software Engineering, \textit{Northeastern University}, China}{2009 -- 2013}

\section{\faUsers\ \textbf{Experience}}

\datedsubsection{\textbf{Postdoctoral Research Fellow (AI4DB)}, \textbf{University of Melbourne}}{2024 -- Present}
\begin{itemize}[parsep=0.35ex]
  \item Lead modeling for spatial index benchmarking: trained RL agents that beat hand-tuned baselines by 18\% throughput while reducing maintenance cost.
  \item Built GenAI copilots for query engineers that summarize execution plans and suggest optimizer hints using GPT-4 + domain finetuning.
  \item Mentor MS students on experimentation hygiene, feature documentation, and reproducibility pipelines.
\end{itemize}

\datedsubsection{\textbf{Data Scientist}, \textbf{\href{https://nftdb.ai/}{nftDb \faExternalLink}}, Remote/Australia}{2023 -- 2024}
\begin{itemize}[parsep=0.35ex]
  \item Modeled anomalous wallet behaviors using PageRank-like graph scores + LightGBM; recall +22\% vs. heuristic baselines over 1.4B transactions.
  \item Designed dbt modeling layers + Looker dashboards that exposed wash-trading KPIs to product and compliance stakeholders weekly.
  \item Shipped GenAI wallet briefing notes (GPT-4 + custom embeddings) that auto-generate collector summaries, adopted by BD and community teams.
\end{itemize}

\datedsubsection{\textbf{Research Assistant / Data Scientist}, \textbf{University of Melbourne}}{2019 -- 2023}
\begin{itemize}[parsep=0.35ex]
  \item Developed a coding-style checker (Python, AST analysis + GenAI repair suggestions) improving student error detection by 35\%.
  \item Instrumented reading-interruption study pipelines: Node.js backend, GCP infra, usage analytics, and Midjourney/GPT content workflows.
\end{itemize}

\datedsubsection{\textbf{Software Engineer}, \textbf{Baidu}, China}{2015 -- 2017}
\begin{itemize}[parsep=0.35ex]
  \item Partnered with data scientists to add telemetry hooks and experimentation switches across the InfoFlow messaging stack (Java, Android).
  \item Implemented on-device ranking features and logging that powered personalization models for 40K+ enterprise users.
\end{itemize}


% \datedsubsection{\textbf{Software Engineer}, Baidu, China}{Jul. 2015 -- Aug. 2017}
% \role{}{Android Development (\textbf{Java})}
% Worked on Instant Messaging (IM) software development, which is an enterprise intelligent platform \href{https://infoflow.baidu.com/}{[\textbf{infoflow}]}. 
% Key contributions are as follows:

% \begin{itemize}
%   \item Designed new message protocols, e.g., Recall message, Delete message, Red pocket Message, etc.
%   \item Designed Voice Assistant module by voice recognition SDK to send message, order meeting room, etc.
%   \item Optimised the database retrieval efficiency by about 20\% by optimising database table index.
%   \item Optimised application centre UI with flexible configuration by new widgets.
%   \item Improved all message module development efficiency by using design patterns, e.g., Builder.
%   \item Analysed, fixed, and documented tough bugs.
% \end{itemize}



% \item Guanli Liu, Jianzhong Qi, Christian S. Jensen, Tianyi Li, Lars Kulik. Efficient Cost Modelling of Space-filling Curves. Submitted to SIGMOD 2024. \href{https://github.com/Liuguanli/LearnSFC}{[\textbf{source code}]} 

% \section{\faFile\ \textbf{Publications}}
% \begin{itemize}
%  \item G. Liu et al. "Efficient Index Learning via Model Reuse and Fine-tuning." \textit{DBML@ICDE 2023}. 
%  \item G. Liu et al. "Efficiently Learning Spatial Indices." \textit{ICDE 2023}.
%  \item J. Qi, G. Liu, et al. "Effectively Learning Spatial Indices." \textit{VLDB}, 2020.
%  \item Y. Gu, G. Liu, et al. "The Moving K Diversified Nearest Neighbor Query." \textit{TKDE}, 2016.
% \end{itemize}
\section{\faFile\ \textbf{Publications \& Impact}}
\begin{itemize}[parsep=0.35ex]
 \item \textbf{VLDB 2024/2025}: Efficient cost modeling of space-filling curves powering adaptive storage layouts.
 \item \textbf{ICDE 2023}: Efficiently learning spatial indexes; toolkit open-sourced and reused by multiple labs.
 \item \textbf{ICDEW 2023}: Model reuse/fine-tuning for learned indexes, highlighting rapid experimentation workflows.
\end{itemize}

\end{document}
