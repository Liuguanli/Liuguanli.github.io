% !TEX program = xelatex

\documentclass{resume}
%\usepackage{zh_CN-Adobefonts_external} % Simplified Chinese Support using external fonts (./fonts/zh_CN-Adobe/)
%\usepackage{zh_CN-Adobefonts_internal} % Simplified Chinese Support using system fonts
\usepackage{xcolor}
\begin{document}
\pagenumbering{gobble} % suppress displaying page number

\name{Guanli Liu}

\basicInfo{
  \email{liuguanli22@gmail.com} \textperiodcentered\ 
  \phone{(+61) 450075953} \textperiodcentered\ 
%  \linkedin[Guanli Liu]{https://www.linkedin.com/in/guanli-liu-11353058/} \textperiodcentered\ 
  \github[https://github.com/Liuguanli]{https://github.com/Liuguanli}}

\section{\faCogs\ Skills}
\begin{itemize}[parsep=0.5ex]
  \item Programming languages: Java, Python, C++, SQL, LaTex
  \item Machine learning tools: TensorFlow, PyTorch, TorchLib, Scikit-learn, Weka,
  \item Research interests: Learned index, Spatial index, Spatial data management

  
%    \item Development: Java Web, Java, Android, SQL, LaTex, TensorFlow, PyTorch
%  \item Development: Java Web, Java, Android, SQL, LaTex, TensorFlow, PyTorch
\end{itemize}

\section{\faGraduationCap\ Education}
\datedsubsection{\textbf{The University of Melbourne}, Melbourne, Australia}{2019 -- Present}
\textit{PhD Candidate} in Computing and Information System, expected \textbf{Jul. 2023} (GPA: H1)
\datedsubsection{\textbf{Northeastern University}, Shenyang, China}{2013 -- 2015}
\textit{M.S.} in Computer Technology (GPA: 82.65 / 100)
\datedsubsection{\textbf{Northeastern University}, Shenyang, China}{2009 -- 2013}
\textit{B.S.} in Software Engineering (GPA: 87.67 / 100)

\section{\faUsers\ Experience}

\datedsubsection{\textbf{The University of Melbourne} Victoria, Australia}{Aug. 2019 -- Present}
\role{Tutor}{Comp90018 (Mobile Computing) \& Comp90041 (Programming and Software Development)} 
%Duties: 
\begin{itemize}
  \item Android (Comp90018) and Java (Comp90041) introduction and demonstration
  \item Project guidance and assignment marking
\end{itemize}

\datedsubsection{\textbf{Baidu Inc.} Beijing, China}{Jul. 2015 -- Aug. 2017}
\role{Senior Researcher and Developer}{Android Development (Java)}
Our team develops enterprise intelligent work platform \href{https://infoflow.baidu.com/}{[\textbf{infoflow}]}, which supports Desktop, iOS, and Android. I work in an Android Development group with around ten engineers, focusing on Instant Messaging (IM) software development.  We pay attention to team work and each single task is finished with PM, UI, UE, QA and interior code review. Key contributions are as follows:
\begin{itemize}
  \item Message design: protocol design, Recall message, Delete message, Red pocket Message, etc
  \item Heartbeat mechanism:
  \item Voice assistant: Voice recognition SDK integration, send message, order meeting room, etc
  \item Application centre: Configure and manage enterprise applications
\end{itemize}



%\role{Tutor}{Comp90041 (Programming and Software Development)}
%%Duties:
%\begin{itemize}
%  \item Java development introduction and questions demonstration
%    \item Assignment marking and feedback
%\end{itemize}

%\datedsubsection{\textbf{Neusoft } Shenyang, China}{Jun. 2012 -- Sep. 2012}
%\role{Software Engineer (internship)}{Java Web development}
%
%\begin{itemize}
%\item Building a logistics management system with Java EE  (Struts + Spring + Hibernate)
%\end{itemize}
  

\section{\faFilesO\ Published Papers}
\begin{itemize}
 \item Guanli Liu, Lars Kulik, Xingjun Ma, Jianzhong Qi. A Lazy Approach for Efficient Index Learning. arXiv preprint arXiv:2102.08081. \href{https://github.com/Liuguanli/ModelReuse}{[\textbf{source code}] (C++)} 
 
  \item Jianzhong Qi (supervisor), Guanli Liu, Christian S. Jensen, and Lars Kulik. Effectively Learning Spatial Indices. PVLDB, 13(11): 2341-2354, 2020.  \href{https://github.com/Liuguanli/RSMI}{[\textbf{source code}] (C++)}

  \item Yu Gu (supervisor), Guanli Liu, Jianzhong Qi, Hongfei Xu, Ge Yu, and Rui Zhang. The Moving K Diversified Nearest Neighbor Query. TKDE, 28(10): 2778-2792, 2016.
\end{itemize}

% Reference Test
%\datedsubsection{\textbf{Paper Title\cite{zaharia2012resilient}}}{May. 2015}
%An xxx optimized for xxx\cite{verma2015large}
%\begin{itemize}
%  \item main contribution
%\end{itemize}



\section{\faHeartO\ Honors and Awards}
\datedline{Yearly performance appraisal in Baidu \textit{Top 20\%}}{2016}
\datedline{Outstanding master student \textit{Top 5\%}}{2014}
\datedline{\textit{\nth{3} Prize} prize of American college students' mathematical modelling contest}{2011}
\datedline{\textit{\nth{3} Prize} prize of Google Android application development in China}{2011}


%\section{\faInfo\ Miscellaneous}
%\begin{itemize}[parsep=0.5ex]
  %\item GitHub: https://github.com/Liuguanli
  %\item Languages: English - Fluent, Mandarin - Native speaker
%\end{itemize}

%% Reference
%\newpage
%\bibliographystyle{IEEETran}
%\bibliography{mycite}
\end{document}
